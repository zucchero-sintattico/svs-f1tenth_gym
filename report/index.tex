\documentclass[conference]{IEEEtran}
% \IEEEoverridecommandlockouts
% The preceding line is only needed to identify funding in the first footnote. If that is unneeded, please comment it out.
\usepackage{cite}
\usepackage{amsmath,amssymb,amsfonts}
\usepackage{algorithmic}
\usepackage{graphicx}
\usepackage{textcomp}
\usepackage{xcolor}

\def\BibTeX{{\rm B\kern-.05em{\sc i\kern-.025em b}\kern-.08em
    T\kern-.1667em\lower.7ex\hbox{E}\kern-.125emX}}
\begin{document}

\title{
    Autonomous driving model trained in a simulated environment using Reinforcement Learning and operating in a ROS environment
}

\author{Manuel Andruccioli,
Tommaso Patriti,
Giacomo Totaro,\\ 
\textit{University of Bologna (Italy)} \\
e-mail: $\{$manuel.andruccioli, tommaso.patriti, giacomo.totato2$\}$@studio.unibo.it }

\maketitle

\begin{abstract}
Our abstract
\end{abstract}

\begin{IEEEkeywords}
    Reinforcement Learning, Deep Learning, Autonomous Racing, ROS
\end{IEEEkeywords}

\section{Introduction}

Section \cite{test}

\begin{itemize}
    \item Descrizione del contesto e dell'importanza della guida autonoma nelle macchine.

    \item Presentazione del vostro obiettivo di ricerca e della vostra ipotesi.

\end{itemize}


\section{Stato dell'arte}

\begin{itemize}
    \item Una revisione della letteratura su progetti simili e sull'uso di Reinforcement Learning nelle applicazioni di guida autonoma.

    \item Discussione delle sfide e delle soluzioni proposte da altri ricercatori nel campo.

\end{itemize}


\section{Metodologia}

\begin{itemize}
    \item Descrizione dell'architettura del modello utilizzato, inclusi i dettagli su come avete implementato l'algoritmo PPO.

    \item Spiegazione del processo di raccolta dei dati e la selezione dei circuiti utilizzati per l'addestramento.

    \item Dettagli sui waypoints e su come sono stati integrati nel processo di addestramento.

\end{itemize}

\section{Esperimenti}

\begin{itemize}
    \item Descrizione delle condizioni sperimentali, tra cui la configurazione dell'addestramento, la scelta dei parametri, ecc.

    \item Presentazione dei risultati ottenuti durante i vostri esperimenti.

    \item Analisi dei risultati, comprese le prestazioni del modello su diversi circuiti.

\end{itemize}

\section{Discussione}

\begin{itemize}
    \item Interpretazione dei risultati e confronto con la letteratura esistente.

    \item Discussione sulle sfide incontrate e le eventuali limitazioni del vostro approccio.

    \item Possibili sviluppi futuri e miglioramenti proposti.

\end{itemize}

\section{Conclusioni}

\begin{itemize}
    \item Riassunto dei risultati principali.

    \item Sottolineare l'importanza del vostro contributo e le potenziali implicazioni nella guida autonoma.

\end{itemize}


\bibliographystyle{IEEEtran}
\bibliography{bibliography}

\end{document}
